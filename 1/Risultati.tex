\section*{Risultati}
In questa sezione sono esposti in primo luogo i risultati delle analisi condotte sui singoli campioni e in seguito le considerazioni prodotte dal confronto tra quest'ultimi. I segnali grezzi ottenuti sono rappresentati in Figura \ref{fig:ir-raw} e in Figura \ref{fig:cpmg-raw}.

\begin{figure}[ht]
\includegraphics[width=\columnwidth]{ir-raw.PNG}
\caption{Segnale IR acquisito, in rosso si ha la parte reale del segnale, in verde la parte immaginaria, in nero il modulo.}
\label{fig:ir-raw}
\end{figure}
\begin{figure}[ht]
\includegraphics[width=\columnwidth]{cpmg-raw.PNG}
\caption{Segnale CPMG acquisito, in rosso si ha la parte reale del segnale, in verde la parte immaginaria, in nero il modulo.}
\label{fig:cpmg-raw}
\end{figure} 

L'analisi dei tempi di rilassamento per i due campioni ha previsto l'elaborazione dei dati attraverso il software UPENWin.

Per valutare con maggior consistenza i risultati il software è stato eseguito sugli stessi dati dei campioni secondo diverse configurazioni.

Le configurazioni impostate per UPENWin riguardo ai dati del tuorlo e dell'albume sono state, eccetto la configurazione di default, l'utilizzo del doppio e del triplo dei punti di inversione per formulare la soluzione quasi-continua o, in alternativa, la variazione del parametro di smoothing. 
Le specifiche delle configurazioni sono citate in seguito. 

Le prime due sezioni sono suddivise in due parti: la prima relativa all'analisi del tempo di rilassamento longitudinale $T_1$, che ha coinvolto la tecnica di Inversion Recovery, e la seconda relativa al tempo di rilassamento trasversale $T_2$, che ha coinvolto la tecnica di CPMG. 



\subsection*{Albume} 

Riguardo all'albume, dall'analisi sul tempo di rilassamento longitudinale $T_1$, i cui valori sono riportati nella prima colonna della tabella \ref{tab:Albume}, sono rilevati due intervalli di tempo i cui segnali non sono trascurabili. 

Tali risultati sono confermati da un'analisi qualitativa dei grafici(\ref{subfig:T_1albume}), vista la presenza simultanea di un picco principale e un secondo picco con intensità di segnale inferiore di almeno due ordini di grandezza. 
In particolare il secondo picco è individuato ad un tempo di rilassamento inferiore, indicando la presenza di un composto diverso dall'albume.

Questo secondo picco è sottolineato nelle configurazioni di oversmoothing di Upen sia per valori del coefficiente di smoothing vicini a 1 che prossimi a 10 e ciò rafforza l'ipotesi della presenza di una piccola impurità all'interno del campione di albume. 

Per quanto riguarda il tempo di rilassamento trasversale $T_2$, il cui valore è stimato nella seconda colonna della tabella \ref{tab:Albume}, sia l'analisi qualitativa dei grafici(\ref{subfig:T_2albume}) che l'analisi quantitativa riportano un comportamento molto simile con $T_1$.

Anche in questo caso è possibile notare dai dati la presenza di un picco più intenso e di un secondo picco a tempi di rilassamento prossimi a 0, che supporta l'ipotesi di impurità nel campione.

Sia per $T_1$ che per $T_2$ la larghezza dei picchi è stretta e ciò ad indicare come i composti siano nelle stesse condizioni. 

\begin{figure}[ht]
\centering
\subfloat[][\emph{$T_1$ albume.}]
	{\label{subfig:T_1albume}
	\includegraphics[width=\columnwidth]{IRalbumeTriplo.png}
	} \quad
\subfloat[][\emph{$T_2$ albume.}]
	{\label{subfig:T_2albume}
	\includegraphics[width=\columnwidth]{CPMGalbume1Triplo.png}
	} \\
\caption{}
\label{fig:T_albume}
\end{figure}

Entrambe le curve nel grafico sono state delineate a partire dai dati generati da UPENWin nella configurazione contenente il triplo dei dati grezzi a disposizione.
\'E stato scelto questo grafico tra tutti quelli ricavati poichè è stato considerato il più rappresentativo.

\begin{table}[ht]
	\centering
	\begin{tabular}{ccc}
	\toprule
					\textbf{Albume}	\\
		$T_1\,(ms)$ 		& 		$T_2\,(ms)$ 		\\	
	\midrule
		$1000\,\pm\,300$	&		$390\,\pm\,110$		\\
		$65\,\pm\,16$		&		$3.9\,\pm\,1.7$		\\
	\bottomrule
	\end{tabular}
	\caption{Stime dei tempi di rilassamento $T_1$ e $T_2$ dell'albume.}
	\label{tab:Albume}
\end{table}

Le stime dei valori dei tempi di rilassamento sono state calcolate mediando i valori relativi al punto con maggiore intensità di segnale tra tutte le configurazioni considerate.

L'errore associato ai valori è stato invece calcolato come la larghezza fra i primi due punti simmetrici dopo il picco, nella configurazione ove tale larghezza è massima.
Conseguentemente le stime dei tempi sono accompagnate da errori molto elevati e ciò è ammissibile vista l'assenza di un metodo generale adeguato per la stima del migliore errore da associare a tale misura.

Studiando la relazione tra i tempi di rilassamento dell'albume, tramite una normalizzazione sui dati, è possibile evidenziare come il tempo di rilassamento $T_2$ sia molto minore rispetto a $T_1$;
ciò a riprova del fatto che il fluido è più viscoso dell'acqua.   
Il grafico seguente \ref{fig:Albume} rappresenta tale evidenza.

\begin{figure}[ht]
\centering
\includegraphics[width=\columnwidth]{T_albume_merge.png}
\caption{Relazione tra la densità di segnale normalizzato e i tempi di rilassamento per il campione di albume.}
\label{fig:Albume}
\end{figure}




\subsection*{Tuorlo}

Le analisi condotte sul tuorlo hanno riportato un andamento differente da quello dell'albume.
I grafici descritti per il tempo di rilassamento $T_1$ contengono un picco pronunciato che per alcune configurazioni ha una coda allungata e per altre si separa da un altro picco meno intenso.
Infatti, per le configurazioni di UPENWin di default e di oversmoothing risulta chiaro quest'ultimo effetto; mentre considerando il doppio e il triplo dei punti di inversione, oltre al caso di undersmoothing, il picco è dotato di una coda allungata verso tempi maggiori.

I dati relativi a $T_1$ si possono trovare nella prima colonna della tabella \ref{tab:Tuorlo} mentre è possibile visualizzare una curva rappresentativa nel sottografico \ref{subfig:T_1tuorlo}. 

Lo studio sul tempo di rilassamento $T_2$ del campione di tuorlo condivide invece alcuni dei risultati che sono stati ottenuti per l'albume. 
Come per quest'ultimo infatti, è presente un picco principale e un abbozzamento a tempi inferiori con intensità nettamente inferiore ma, come per il $T_1$, per valori di tempi maggiori del picco principale la curva descrive un andamento più incerto.
Infatti, come nel caso precedente per alcune configurazioni si manifesta un altro picco e per altre un allungamento della coda.

I valori relativi a $T_2$ per i due picchi sono segnati nella seconda colonna della tabella \ref{tab:Tuorlo}; mentre il sottografico \ref{subfig:T_2tuorlo} delinea l'andamento più consistente.

\begin{figure}[ht]
\centering
\subfloat[][\emph{$T_1$ tuorlo.}]
	{\label{subfig:T_1tuorlo}
	\includegraphics[width=\columnwidth]{IRtuorloTriplo.png}
	} \quad
\subfloat[][\emph{$T_2$ tuorlo.}]
	{\label{subfig:T_2tuorlo}
	\includegraphics[width=\columnwidth]{CPMGtuorlo1Triplo.png}
	} \\
\caption{}
\label{fig:T_tuorlo}
\end{figure}

Come per il campione d'albume anche per il tuorlo i grafici considerati sono stati derivati dalla configurazione di UPENWin che elabora il triplo dei dati a disposizione.

\begin{table}[ht]
	\centering
	\begin{tabular}{ccc}
	\toprule
					\textbf{Tuorlo}	\\
		$T_1\,(ms)$ 		& 		$T_2\,(ms)$ 		\\	
	\midrule
		$46\,\pm\,13$		&		$16\,\pm\,4$		\\
		$100\,\pm\,40$		&		$80\,\pm\,30$		\\
	\bottomrule
	\end{tabular}
	\caption{Stime dei tempi di rilassamento $T_1$ e $T_2$ del tuorlo.}	
	\label{tab:Tuorlo}
\end{table}

Come già citato precendemente le stime sono state calcolate come media tra le varie configurazione dei valori con maggior intensità e gli errori come la larghezza fra i primi due punti simmetrici dopo il picco, nella configurazione ove tale larghezza è massima.

La relazione tra i tempi di rilassamento nel tuorlo mostrano anche in questo caso che il tempo $T_2$ è molto minore del tempo $T_1$. 
Tale risultato è meglio espresso dal grafico seguente \ref{fig:Tuorlo}.

\begin{figure}[ht]
\centering
\includegraphics[width=\columnwidth]{T_tuorlo_merge.png}
\caption{Relazione tra la densità di segnale normalizzato e i tempi di rilassamento per il campione di tuorlo.}
\label{fig:Tuorlo}
\end{figure}


\section*{Confronto tra campioni}

La relazione tra i due campioni rivela come è possibile discriminare i due composti principali dell'uovo e come tale discriminazione può essere evidente.

Considerando i tempi di rilassamento longitudinali dal grafico \ref{subfig:T_1} è possibile osservare che il tempo associato all'albume è molto maggiore rispetto a quello misurato per il tuorlo. 
Lo stesso risultato è ottenuto dal confronto tra i tempi di rilassamento trasversali, figura \ref{subfig:T_2}.

\begin{figure}[ht]
\centering
\subfloat[][\emph{Confronto tra i tempi di rilassamento longitudinali.}]
	{\label{subfig:T_1}
	\includegraphics[width=\columnwidth]{T_1_merge.png}
	} \quad
\subfloat[][\emph{Confronto tra i tempi di rilassamento trasversali.}]
	{\label{subfig:T_2}
	\includegraphics[width=\columnwidth]{T_2_merge.png}
	} \\
\caption{}
\label{fig:Confronto}
\end{figure}

