\section*{Introduzione}
L'esperimento effettuato si è basato sui principi della risonanza magnetica nucleare ed è stato realizzato secondo specifici metodi di acquisizione. 

La risonanza magnetica nucleare è un fenomeno fisico che comporta la transizione tra stati energetici del momento angolare nucleare dei nuclei atomici.
Il fenomeno si verifica solo in presenza di nuclei atomici caratterizzati da momento angolare, che permette l'interazione con campi magnetici esterni.
La combinazione delle interazioni dei vari nuclei con lo stesso campo magnetico induce la formazione di una magnetizzazione nucleare di equilibrio, da cui è possibile ricavare informazioni sul sistema studiato.

Infatti, le informazioni si rilevano studiando il ritorno della magnetizzazione allo stato di equilibrio.
Ciò è realizzato cedendo al sistema energia in condizione di risonanza e lasciando che la magnetizzazione tenda a riportarsi allo stato d'ordine iniziale. 

All'equilibrio, la magnetizzazione nucleare è allineata con il campo magnetico uniforme applicato, detto $B_0$, e, una volta stimolato il sistema applicando un impulso la magnetizzazione è caratterizzata da un moto.
L'impulso esercitato equivale ad un campo magnetico stimolante, detto $B_1$, che è perpendicolare a $B_0$ ed oscillante con una frequenza pari alla frequenza di risonanza. 
L'applicazione di $B_1$ allontana la magnetizzazione dallo stato stazionario e innesca un moto per la magnetizzazione.
Il moto innescato è una combinazione di rotazioni sia attorno a $B_0$ che alla direzione di $B_1$.
La rotazione per $B_1$ permette di modificare la direzione della magnetizzazione e quindi il tempo di applicazione dell'impulso incide sulla grandezza dell'angolo di rotazione, chiamato flip angle.

Al termine dell'impulso, se il sistema si trovasse in condizioni ideali, la magnetizzazione dovrebbe continuare solo a ruotare attorno a $B_0$ alla frequenza di risonanza. 
Ciò però non si verifica a causa delle diverse posizioni degli atomi nel sistema che comporta la percezione della non perfetta uniformità campo magnetico $B_0$. 
In conseguenza gli atomi sono caratterizzati da diverse frequenze di rotazione dei singoli momenti angolari che comporta una dispersione del vettore magnetizzazione sul piano perpendicolare alla direzione di $B_0$.
In questo modo il vettore magnetizzazione ritorna alla posizione di equilibrio e questo effetto è denominato rilassamento.

Il rilassamento è studiato in due componenti, la componente lungo la direzione del campo magnetico $B_0$ (a cui associamo l'asse $z$) e la componente sul piano perpendicolare (piano $xy$). 
Entrambe le componenti sono perciò caratterizzate da un relativo tempo di rilassamento, che nel caso della componente $z$ chiameremo $T_1$ e nel caso del piano $xy$ chiameremo $T_2$.
A partire dallo studio di questi due tempi di rilassamento, sono state sviluppate tecniche per l'elaborazione di immagini MRI.

Nel prossimo paragrafo saranno presentati i metodi e la strumentazione adoperati in questo esperimento per la misurazione di queste due grandezze.
