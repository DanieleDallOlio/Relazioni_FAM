\section*{Conclusioni}
I risultati ottenuti hanno confermato le ipotesi qualitative attese.
Infatti, dai grafici \ref{fig:D_h2o}, \ref{fig:D_s130} e \ref{fig:D_s170} è possibile osservare una traslazione dei picchi verso l'origine degli assi in relazione all'incremento di $\tau$.
Analogamente dai grafici \ref{fig:S_h2o}, \ref{fig:S_s130} e \ref{fig:S_s170} è confermato l'aumento dell'inclinazione delle rette per $\tau$ crescente. 

La tabella seguente \ref{tab:final} riassume i valori ricavati per i coefficienti di diffusione delle sostanze utilizzate. 

\begin{table}[h!]
    \begin{center}
    \begin{tabular}{c c c}
    \toprule
    	& $D\,(\frac{{\mu}^2}{ms})$ & $\sigma\,(\frac{{\mu}^2}{ms})$ \\
    \midrule
    	$H_2O$ & 2.04	&	0.16	\\
    	\text{Soltrol 130} & 0.927	&	0.015	\\
    	\text{Soltrol 170} & 0.425	&	0.011	\\
    \bottomrule
    \end{tabular}
    \caption{Coefficienti di autodiffusione per ogni sostanza.}
    \label{tab:final}
    \end{center}
\end{table}

\'E possibile concludere dalla tabella che l'acqua è la sostanza che diffonde più rapidamente.

Il valore del coefficiente di autodiffusione riportato per l'acqua è però differente dal dato teorico conosciuto, pari a $3\,\frac{{\mu}^2}{ms}$, e seppur effetto da errore non è presente interpolazione.

Trascurando possibili errori sistematici sconosciuti e le piccole impurità nel campione, la netta differenza tra valore teorico e dato sperimentale è dovuta alla diffusione ristretta.

Infatti, quando una sostanza è racchiusa in un volume sufficientemente piccolo la diffusione risente della restrizione spaziale comportando un abbattimento del valore del coefficiente di autodiffusione.
In particolare, le molecole dell'acqua sbattendo contro le pareti del contenitore si ritrovano ad occupare posizioni sempre più ristrette e influenzate dalla forma della superficie di contenimento.

Il sistema di acquisizione usato non riconosce l'assenza di diffusione libera e ciò induce al calcolo di un coefficiente di autodiffusione minore.

Per questa ragione, in questo esperimento, il coefficiente di autodiffusione dell'acqua è attenuato di un terzo del valore teorico.

 
