\maketitle

\begin{abstract}
Si misura il coefficiente di autodiffusione molecolare di tre liquidi idrogenati (Acqua, Soltrol 130, Soltrol 170) tramite analisi dell'attenuazione dell'eco di spin in una regione con gradiente di campo magnetico. Si fa uso di un MObile Universal Surface Exploer per la creazione del gradiente di campo magnetico e la cattura dei dati tramite sequenze CPMG ed infine si utilizza il software UpenWIN per la stima delle frequenze misurate.
\end{abstract}

\section*{Introduzione}

In questa parte introduttiva si illustra brevenente la teoria dietro i processi di diffusione molecolare ed in che modo la risonanza magnetica nucleare si colloca come strumento di analisi molto valido \cite{website}.

La diffusione è il moto traslazionale stocastico di molecole e ioni all'interno di una soluzione. Una molecola costituente di un fluido è soggetta a moto browniano. Se il fluido in cui è immersa è omogeneo isotropo ed infinito, la molecola è soggetta a diffusione isotropa tale per cui vale lo spostamento quadratico medio:
\begin{equation}
	<x^2(t)> = (2Dt)
\end{equation}
dove $D$ prende il nome di costante di diffusione e si collega direttamente alle dimensioni della molecola secondo l'equazione di Stokes-Einsten,
\begin{equation}
	D = KT/f
\end{equation}
dove, nel caso di molecola sferica, $f = 6\pi\mu r$.

Se la specie molecolare del soluto è la stessa del mezzo in cui le molecole del soluto diffondono, si definisce il tutto un processo di \textbf{auto-diffusione}.

Con l'NMR è possibile studiare la diffusione traslazionale basandosi sull'attenuazione dell'eco di spin causato dallo sfasamento degli spin nucleari dovuto ai processi di diffusione in un una regione con gradiente di campo magnetico $g$ costante.

Considerando un gradiente $g$ diretto verso $z$ ed una singola molecola dotata di spin, si ha che la fase accumulata nel tempo dalla molecola è data da:
\begin{equation}
	\Phi(t) = \gamma B_0 t + \gamma \int_0^t g(t')z(t')\,dt'
\end{equation}
dove la prima parte è data dal campo statico e la seconda dal gradiente applicato. Se il gradiente $g(t)$ ha intensità costante l'equazione si semplifica in:
\begin{equation}
	\Phi_i(\tau) = \gamma B_0 \tau + \gamma g \int_{0}^{\tau}z_i(t')\,dt'
\end{equation}

Se, dopo il tempo $\tau$, mandiamo alla molecola un impulso $\pi$ di inversione, abbiamo che il secondo impulso di gradiente mandato al tempo $t_0 + \tau$, otteniamo al tempo di eco una fase accumulata:
\begin{multline}
	\Phi(2\tau) = \left\{ \gamma B_0 \tau + \gamma g \int_{t_0}^{t_0 + \tau}z_i(t')\,dt'\right\}_{\text{primo periodo } \tau} \\- \left\{ \gamma B_0 \tau + \gamma g \int_{t_0 + \tau}^{t_0 + 2\tau}z_i(t')\,dt'\right\}_{\text{secondo periodo } \tau}
	\\
	= \gamma g \left\{ \int_{t_0}^{t_0+\tau}z_i(t')\,dt' - \int_{t_0 + \tau}^{t_0 + 2\tau}z_i(t')\,dt' \right\}
\end{multline}

Ergo, per un insieme di nuclei con differenti posizioni iniziali e finali, si avrà un segnale di eco in forma di:
\begin{equation}
	S(2\tau) = S(2\tau)_{g=0}\int_{-\infty}^{+\infty} P(\Phi,2\tau)e^{i\Phi}\,d\Phi
\end{equation}
dove $P(\Phi,2\tau)$ è la probabilità che il singolo spin abbia accumulato la fase $\Phi$. Questo implica che con assenza di gradiente abbiamo segnale massimo (considerandone solo la parte reale), mentre invece con presenza di gradiente si osserva un segnale attenuato.

Si ha quindi l'espressione:
\begin{equation}
	S(2\tau) = S(0) \exp\left(-\frac{2\tau}{T_2}\right) f(\delta,g,\Delta,D)
\end{equation}

che, per una sequenza SE in condizione di gradiente costante, assume la forma:
\begin{equation}
	S(2\tau) = S(0)\exp(-\frac{2\tau}{T_2})\exp(-\frac{2}{3}\gamma^2Dg^2\tau^3)
\end{equation}
mentre invece, per una sequenza CPMG in condizione di gradiente costante, assume la forma:
\begin{equation}
	S(t) = S(0)\exp\left[-\frac{t}{T_2}-\frac{1}{3}\gamma^2 g^2 D\tau^2 t\right]
\end{equation}
dove, in questo contesto, $\tau$ rappresenta metà del tempo di eco della sequenza CPMG.

A questo punto, se si vuole misurare il valore di $D$ tramite misure CPMG, si può operare sulla variazione del tempo di eco $2\tau$ e fare uso della funzione:
\begin{equation}
	R_{\text{2oss}}(\tau) = \frac{1}{T}_{\text{2oss}}(\tau)=\frac{1}{T_2}+\frac{1}{3}\gamma^2g^2D\tau^2
\label{eq:R}
\end{equation}
dove si distingue $T_{\text{2oss}}$ come valore osservato in contrapposizione al $T_2$ ideale non affetto da diffusione. Da questa equazione figura come dalla dipendenza lineare di $R_{\text{2oss}}(\tau)$ da $D$ è possibile, se si osservano andamenti lineari, eseguire il best fit dei dati sperimentali ad una retta ed ottenere il valore della costante di diffusione.

